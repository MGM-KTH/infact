%%%%%%%%%%%%%%%%%%%%%%%%%%%%%%%%%%%%%%%%%
% Short Sectioned Assignment
% LaTeX Template
% Version 1.0 (5/5/12)
%
% This template has been downloaded from:
% http://www.LaTeXTemplates.com
%
% Original author:
% Frits Wenneker (http://www.howtotex.com)
%
% License:
% CC BY-NC-SA 3.0 (http://creativecommons.org/licenses/by-nc-sa/3.0/)
%
%%%%%%%%%%%%%%%%%%%%%%%%%%%%%%%%%%%%%%%%%

%----------------------------------------------------------------------------------------
%	PACKAGES AND OTHER DOCUMENT CONFIGURATIONS
%----------------------------------------------------------------------------------------

\documentclass[paper=a4, fontsize=11pt,numbers=endperiod]{scrartcl} % A4 paper and 11pt font size

\usepackage[T1]{fontenc} % Use 8-bit encoding that has 256 glyphs
%\usepackage{fourier} % Use the Adobe Utopia font for the document - comment this line today return to the LaTeX default
\usepackage[english]{babel} % English language/hyphenation
\usepackage{amsmath,amsfonts,amsthm} % Math packages

\usepackage[utf8]{inputenc} % Needed to support swedish "åäö" chars
\usepackage{titling} % Used to re-style maketitle

\usepackage{lipsum} % Used for inserting dummy 'Lorem ipsum' text into the template

\usepackage{sectsty} % Allows customizing section commands
\allsectionsfont{\normalfont} % Make all sections the default font

% Local packages:
\usepackage{enumerate}
\usepackage[usenames,dvipsnames]{color}
\usepackage{tabularx}
\usepackage{fancyvrb}
\DefineShortVerb{\|}
\usepackage{hyperref}
\usepackage{url}
\usepackage[parfill]{parskip}   % Sets newlines between paragraphs
\usepackage{algorithm2e} % Used for Pseudocode
\providecommand{\abs}[1]{\lvert#1\rvert} % \abs{x+y} produces |x+y|



\usepackage{fancyhdr} % Custom headers and footers
\pagestyle{fancyplain} % Makes all pages in the document conform to the custom headers and footers

% Header with additional info
% \fancyhead[L]{\small{Gustaf Lindstedt - \href{mailto:glindste@kth.se}{\color{RoyalBlue}\nolinkurl{glindste@kth.se}} - 910301\\Martin Runelöv - \href{mailto:mrunelov@kth.se}{\color{RoyalBlue}\nolinkurl{mrunelov@kth.se}} - 900330-5738}}
% Simple header
\fancyhead[L]{\small{Gustaf Lindstedt\\Martin Runelöv}} % 


\fancyfoot[L]{} % Empty left footer
\fancyfoot[C]{} % Empty center footer
\fancyfoot[R]{\thepage} % Page numbering for right footer
\renewcommand{\headrulewidth}{0pt} % Remove header underlines
\renewcommand{\footrulewidth}{0pt} % Remove footer underlines
\setlength{\headheight}{23.0pt} % Customize the height of the header
\fancyhfoffset[L]{10mm}% slightly less than 0.25in

\numberwithin{equation}{section} % Number equations within sections (i.e. 1.1, 1.2, 2.1, 2.2 instead of 1, 2, 3, 4)
\numberwithin{figure}{section} % Number figures within sections (i.e. 1.1, 1.2, 2.1, 2.2 instead of 1, 2, 3, 4)
\numberwithin{table}{section} % Number tables within sections (i.e. 1.1, 1.2, 2.1, 2.2 instead of 1, 2, 3, 4)

\setlength\parindent{0pt} % Removes all indentation from paragraphs - comment this line for an assignment with lots of text


\posttitle{\par\end{center}} % Remove space between author and title
%----------------------------------------------------------------------------------------
%	TITLE SECTION
%----------------------------------------------------------------------------------------

\title{	
\huge Project 1 - Factoring \\ % The assignment title
\vspace{10pt}
\normalfont \normalsize 
\textsc{DD2440 - Advanced algorithms } \\ [25pt] % 
}

\author{\vspace{-20pt} Gustaf Lindstedt - \nolinkurl{glindste@kth.se} - 910301\\\\\\Martin Runelöv - \nolinkurl{mrunelov@kth.se} - 900330-5738}

\date{\vspace{8pt}\normalsize\today} % Today's date or a custom date

\begin{document}

\maketitle % Print the title

%-------------------------------------------------------------------------------
%	SECTION 1
%-------------------------------------------------------------------------------

\section{Introduction}

\subsection{Problem}
A prime number is a positive integer greater than 1 that has no positive integer divisors other 1 and itself. The prime factorization of a positive integer is a list of primes that divides the integer, and their multiplicities. The problem of finding these prime factors is called integer factorization.

The fundamental theorem of arithmetic states that every positive integer (> 1) is either prime or a unique product of prime numbers. In other words, all natural numbers can be factorized.



%-------------------------------------------------------------------------------
%	SECTION 2
%-------------------------------------------------------------------------------
\section{Method}
The factorization program was written in |C| and uses the library GMP\cite{gmp} to represent large numbers, perform functions such as primality testing and GCD.

The first algorithm used was Pollard's rho algorithm (PR). A static list of the first 10 000 primes was used to weed out trivial primes with trial division. When using PR only the first 500-2000 primes are used.

Before each iteration of PR, primality testing is performed. This is done using GMP's built-in function |mpz_probab_prime| which uses trial division and Miller-Rabin probabilistic primality tests internally.\cite{probabprime}


In the pseudocode below, the pseudo-random generator $f(x)$ is defined as: $x^2+C\, (mod\, N)$, where $C$ is an arbitrary constant which maintains the property $x = y\, (mod\; p) \rightarrow f(x) = f(y)\,(mod\; p)$.\\

\begin{algorithm}[H]
 \SetAlgoLined % For previous releases [?]
 \textbf{Input:} {A composite number N}\\
 \textbf{Output:} {A non-trivial factor of N}\\
 let $x_0 \in_U \mathbb{Z}_N$\\
 $x = x_0$\\
 $y = x_0$\\
 \While{True}{
  $x := f(x)$\;
  $y := f(f(y))$\;
  $d := gcd( \abs{y-x}, N)$\;
  \If{$d > 1$}{
   return d\;
   }
 }
 \caption{Pseudocode for Pollard's rho algorithm}
\end{algorithm}

\hspace{0pt}\\\\
In practice, the $while$-loop runs a finite number of times, since Pollard's rho algorithm can fail to find factors, and since the program has a deadline to keep.

Pollard's rho algorithm (PR) \cite{pollard}


Brent's factorization method (BFM) \cite{brent} 

Another paper that describes several algorithms, including Brent (not the version we're using, this one gave $\sim$57 in Kattis. A lot simpler though.)\cite{otherpaper}


\subsection{Alternative methods}

The quadratic sieve (QS) \cite{qsieve}

Fermat's factorization method (FFM) \cite{fermat}

%-------------------------------------------------------------------------------
%   SECTION 3
%-------------------------------------------------------------------------------

\section{Results}


    \begin{tabular}{|c|c|c|c|c|}
    \hline
    \textbf{Algorithm} & \textbf{Factorized numbers} \\ \hline
    Trial division & 11 \\ \hline
    Trial division + mpz\_probab\_prime & 19 \\ \hline
    Pollard's rho & 73 \\ \hline
    Brent's & 84 \\ \hline
    \end{tabular}
    \hspace{10pt}

\begin{thebibliography}{9}

\bibitem{gmp}\url{http://gmplib.org} <insert date here!>
\bibitem{probabprime}\url{http://gmplib.org/manual/Number-Theoretic-Functions.html} <insert date here!>
\bibitem{pollard}\url{http://en.wikipedia.org/wiki/Pollard's_rho_algorithm} <insert date here!>
\bibitem{brent}\url{http://maths-people.anu.edu.au/~brent/pub/pub051.html} <insert date here!>
\bibitem{otherpaper}\url{http://citeseerx.ist.psu.edu/viewdoc/download?doi=10.1.1.117.1230&rep=rep1&type=pdf} <insert date here!>
\bibitem{qsieve}\url{http://en.wikipedia.org/wiki/Quadratic_sieve} <insert date here!>
\bibitem{fermat}\url{http://en.wikipedia.org/wiki/Fermat's_factorization_method} <insert date here!>
\end{thebibliography}

\end{document}



